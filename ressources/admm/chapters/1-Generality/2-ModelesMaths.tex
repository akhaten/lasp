\subsection{Modèles mathématiques}

\frame{
    \tableofcontents[ 
        currentsubsection, 
    ]
}

\begin{frame}{Modèle mathématique générique}
% \begin{align*}
% \hat{f} = \arg\min\limits_{f} \{ 
%     \frac{\alpha}{2} 
%     \lVert g - SHf \rVert_{2}^{2}
%     + \sum\limits_{i \in \Omega} \beta_{i} \times R_{i}(f)
% \}  
% \end{align*}
\begin{align*}
\hat{f} = \arg\min\limits_{f} \{ 
    \frac{\alpha}{2} 
    \underbrace{\lVert g - SHf \rVert_{2}^{2}}_{\text{Attache aux données}} 
    + \underbrace{
        \sum\limits_{i \in \Omega} \beta_{i} \times R_{i}(f)
    }_{\text{combinaison de régularisation}}
\} 
\end{align*}
Avec, \begin{itemize}
\item $g$ : la donnée observée avec détérioration
\item $S$ : un opérateur de décimation
% \item $h$ un kernel de convolution 
\item $H$ : la matrice BCCB obtenue à partir du kernel $h$
\item $f$ : la donnée désirée sans détérioration
\item $\alpha$ : un hyper-paramètre
\item $\Omega$ : l'ensemble des indices des régularisations
\item $\beta_{i}$ : l'hyper-paramètre associé au terme 
    de régularisation $R_{i}$
\item $R_{i}$ : régurisation d'indice $i$ \\
\end{itemize}
\end{frame}

\begin{frame}{Modèle mathématique générique}
\begin{align*}
g &= S(h \circledast f) + n \\
&= S H f + n \iff n = g - S H f
\end{align*}
\hrule
\begin{align*}
\hat{f} = \arg\min\limits_{f} \{ 
    \frac{\alpha}{2} 
    \lVert g - SHf \rVert_{2}^{2}
    + \sum\limits_{i \in \Omega} \beta_{i} \times R_{i}(f)
\}  
\end{align*}
% \begin{align*}
% \hat{f} = \arg\min\limits_{f} \{ 
%     \frac{\alpha}{2} 
%     \underbrace{\lVert g - SHf \rVert_{2}^{2}}_{\text{Attache aux données}} 
%     + \underbrace{
%         \sum\limits_{i \in \Omega} \beta_{i} \times R_{i}(f)
%     }_{\text{combinaison de régularisation}}
% \} 
% \end{align*}
\end{frame}

\begin{frame}{Modèle mathématique générique}
\begin{align*}
\hat{f} = \arg\min\limits_{f} \{ 
    \frac{\alpha}{2} 
    \lVert g - SHf \rVert_{2}^{2}
    + \sum\limits_{i \in \Omega} \beta_{i} \times R_{i}(f)
\}  
\end{align*}
Dans un langage plus naturel :
\begin{center}
\textbf{
    \emph{
        Nous recherchons $f$ tel qu'il minimise le bruit $n$ \\
        sachant que nous avons un ensemble de 
        connaissances/régularisations \\
        certaines sur $f$. \\
    }
}
\end{center} 
\end{frame}

% \subsection{Modèle inspiré de Mumford-Shah pour la super-résolution}

% \frame{
%     \tableofcontents[ 
%         currentsubsection, 
%     ]
% }


