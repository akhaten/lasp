\subsection{Problématique}

\frame{
    \tableofcontents[ 
        currentsubsection, 
        % hideothersubsections,
        % sectionstyle=show/hide,
        % subsectionstyle=show/shaded,
    ]
}


\begin{frame}{Présentation de la problématique}
\begin{center}
\textbf{
    \emph{
        "Comment pouvons nous retrouver une donnée originale sachant \\
        qu'elle a été détérioré ?"
    }
    % décimée, bruitée et floutée
}
\end{center} 
\end{frame}

\begin{frame}{Vers un problème formel}
\begin{align*}
g &= S(h \circledast f) + n \\
&= S H f + n
\end{align*}

Avec, \begin{itemize}
\item $g$ l'image observée avec une basse résolution
\item $S$ est un opérateur de décimation
\item $h$ un kernel de convolution 
\item $H$ la matrice BCCB obtenue à partir du kernel $h$
\item $f$ l'image désirée avec une haute résolution
\item $n$ un bruit additif
\item $\circledast$ l'opération de la convolution circulaire \\
\end{itemize}
\end{frame}


\begin{frame}{Vers un problème formel}
\begin{align*}
g &= S(h \circledast f) + n \\
&= S H f + n
\end{align*}

L'énoncé formalisé se lit comme : \\
\begin{center}
\textbf{
    \emph{
        $g$ est le résultat de $f$ ayant subit un flou "$H$" suivi \\
        d'une décimation "$S$", puis bruité par "$n$". \\
        L'objectif est donc de retrouver $f$. \\
    }
}
\end{center} 
\end{frame}


\begin{frame}{Les problèmes inverses}
\begin{itemize}
\item C'est un problème inverse
\item Dans la littérature, ce genre de problème est abordé avec
    des modèles mathématiques
\item Il existe des algorithmes itératifs permettant de résoudre
    des problèmes inverses en minimisant un modèle mathématique
\end{itemize}
\end{frame}
    