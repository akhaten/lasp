\subsection{Exemple}

\frame{
    \tableofcontents[ 
        currentsubsection, 
        % hideothersubsections,
        % sectionstyle=show/hide,
        % subsectionstyle=show/shaded,
    ]
}

% \begin{frame}{Modèle inspiré de Mumford-Shah}
% En utilisant les deux "à prioris" (termes de régularisation) ci-dessous
% \begin{align*}
% \begin{dcases*}
% R_{0}(f) = \frac{1}{2} \lVert \nabla f \rVert_{2}^{2} \\
% R_{1}(f) = \lVert \nabla f \rVert_{1}
% \end{dcases*}
% \end{align*}
    
% nous avons la formulation suivante :
% \begin{align*}
% \hat{f} = \arg\min\limits_{f} \{ 
% \frac{\alpha}{2} 
% \underbrace{\lVert g - SHf \rVert_{2}^{2}}_{\text{Attache aux données}} 
% + \frac{\beta_{0}}{2}
% \underbrace{\lVert \nabla f \rVert_{2}^{2}}_{\text{Energie de Dirichlet}}
% + \beta_{1} 
% \underbrace{\lVert \nabla f \rVert_{1}}_{\text{Variation Totale}}
% \}  
% \end{align*}
% \end{frame}

\begin{frame}{Tikhonov : $R(x) = \lVert x \rVert_{2}^{2}$}

\begin{align*}
&\hat{x} = \arg\min\limits_{x} \{ 
    \frac{\alpha}{2} \lVert y - Hx \rVert_{2}^{2} 
    + \beta_{0} \lVert x \rVert_{2}^{2}
\} \\
&\frac{d}{dx} \{
    \frac{\alpha}{2} \lVert y - Hx \rVert_{2}^{2} 
    + \beta_{0} \lVert x \rVert_{2}^{2}
\} = 0 \\
&\iff - \alpha H^{T} (y - Hx) + 2 \beta_{0} x = 0 \\
&\iff - \alpha H^{T} y + \alpha H^{T}Hx + 2 \beta_{0} x = 0 \\
&\iff - \alpha H^{T} y + (\alpha H^{T}H + 2 \beta_{0} I) x = 0 \\
&\iff x = (\alpha H^{T}H + 2\beta_{0} I)^{-1} \alpha H^{T} y \\
\end{align*}

\end{frame}