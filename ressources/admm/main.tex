\documentclass[aspectratio=43]{beamer}
\usepackage[english]{babel}
% \usepackage[linesnumbered, ruled, vlined]{algorithm2e}
\usepackage{algorithm,algpseudocode}
\input{chapters/preamble}
\title{
    Optimization problem
} %->->->->-> Check hyperref title <-<-<-<-<-
% \subtitle{And Some Things About It}
\author{
    \href{https://github.com/akhaten}{Jessy Khafif}
}

% \institute[UT3]{
%     UT3 Paul Sabatier%
%     % \\%
%     % University of São Paulo%
% } %You can change the Institution if you are from somewhere else
% % \date{February 31, 2019}
% % \logo{\includegraphics[width= 0.3\textwidth]{images/ut3_logo.png}}

\begin{document}
    
    \frame{\titlepage}
    
    \begin{frame}{Summary}
        \tableofcontents
    \end{frame}
    
    % \input{chapters/a-silly-idea.tex} %You can put the frames directly into the presentation, but using the input command and writing them in separate .tex files might be more organized

    % \section{Generality}
\frame{\sectionpage}

% \begin{frame}{Défintion}
% \begin{equation*}
% \hat{a} = \arg \min_{a} \{ \frac{1}{2} \lVert b - a \rVert_{2}^{2} + \lambda P(a) \}
% \end{equation*}
% \begin{enumerate}
%     \item $a$ est une image non observable
%     \item $b$ est l'image observable depuis laquelle pour estimer $a$
%     \item $\hat{a}$ est l'image estimée
%     \item $P(a)$ est un "à priori" sur $a$
% \end{enumerate}
% \vspace{3mm}
% On cherche une estimation de $a$ tel que la différence entre $a$ et $b$
% soit minimale et sachant que l'on a une connaissance $P(a)$ sur l'image à estimer.
% \end{frame}

% \begin{frame}{À priori pour le débruitage}
% \begin{enumerate}
%     \item $P(a) = \lVert a \rVert_{1}$
%     \item $P(a) = \lVert a \rVert_{2}^{2}$
%     \item $P(a) = \lVert \nabla a \rVert_{1}$
% \end{enumerate}
% \end{frame}

% \begin{frame}{Débruitage}
% Formulation d'un problème de débruitage:
% \begin{equation*}
% y = Hx + n
% \end{equation*}
% \begin{enumerate}
%     \item $y$ est un signal bruité de dimension $NM$
%     \item $H$ est une matrice BCCB de dimension $NM \times NM$
%     \item $x$ est le signal à retrouver de dimension $NM$
%     \item $n$ est un bruit additif blanc gaussien de dimension $NM$
% \end{enumerate}
% \end{frame}

% \begin{frame}{À priori pour le débruitage}
% Le débruitage consiste à minimiser le bruit.
% \begin{align*}
% y &= Hx + n \\
% \iff n &= y - Hx
% \end{align*}
% À l'aide d'un opérateur proximal, on a \textbf{le problème général d'optimisation suivant} :
% \begin{align*}
% \hat{x} &= \arg \min_{x} \{ \frac{1}{2} \lVert n \rVert_{2}^{2} + \lambda P(x) \} \\
% &= \arg \min_{x} \{ \frac{1}{2} \lVert y - Hx \rVert_{2}^{2} + \lambda P(x) \}
% \end{align*}
% \end{frame}

% \begin{frame}{$P(x) = \lVert \nabla x \rVert_{1}$}
% \begin{align}
% \hat{y} &= \arg \min_{x} \{ \frac{1}{2} \lVert n \rVert_{2}^{2} + \lambda P(x) \} \\
% &= \arg \min_{x} \{ \frac{1}{2} \lVert y - Hx \rVert_{2}^{2} + \lambda P(x) \}
% \end{align}
% \end{frame}

\begin{frame}{Débruitage avec un à priori P(x)}

Opérateur proximal : $\hat{x} = \arg\min\limits_{x} \{ \frac{1}{2} \lVert y - Hx \rVert_{2}^{2} + \lambda P(x)$ \} \\

\vspace{5mm}

Résolution avec l'ADMM:
\begin{itemize}
    \item Équation : $\min\limits_{x, z} \{ \frac{1}{2} \lVert y - Hx \rVert_{2}^{2} + \lambda P(z) \}$
    \item Contrainte : ...
\end{itemize}

\vspace{5mm}

Lagrangien augmenté :

\begin{align*}
L(x, z, u) 
&= \frac{1}{2} \lVert y - Hx \rVert_{2}^{2} + \lambda P(z) +
u^{T}(x-z) + \frac{\rho}{2} \lVert x-z \rVert_{2}^{2} \\
&= \frac{1}{2} \lVert y - Hx \rVert_{2}^{2} + \lambda P(z) +
\frac{\rho}{2} \lVert x-z+\frac{u}{\rho} \rVert_{2}^{2} - \frac{\lVert u \rVert_{2}^{2}}{2\rho}    
\end{align*}

\end{frame}

\begin{frame}{Algorithm}

\begin{algorithm}[H]
    \caption{ADMM} % \label{euclid}
    \begin{algorithmic}[1]
        \Procedure{ADMM}{}\newline
        \textbf{Input:} $y$, $\lambda$, $\rho$, $nb\_iterations$ \\
        \textbf{Output:} $x_{nb\_iterations}$
        \For{\texttt{$k \in 0...nb\_iterations$}}
            \State{$x_{k+1} = \arg\min\limits_{x} {L(x, z_{k}, u_{k})}$}
            \State{$z_{k+1} = \arg\min\limits_{z} {L(x_{k+1}, z, u_{k})}$}
            \State{$u_{k+1} = u_{k} + \rho (z_{k+1} - x_{k+1})$}
        \EndFor
        \EndProcedure
    \end{algorithmic}
    % \label{alg_1}
\end{algorithm}
\end{frame}

\begin{frame}{$\hat x = \arg\min\limits_{x} {L(x, z, u)}$}
    \begin{align*}
        \hat x
        &= \arg\min\limits_{x} L(x, z, u) \\
        &= \arg\min\limits_{x} \{
            \frac{1}{2} \lVert y - Hx \rVert_{2}^{2} + \lambda P(z) +
            \frac{\rho}{2} \lVert x-z+\frac{u}{\rho} \rVert_{2}^{2} 
            - \frac{\lVert u \rVert_{2}^{2}}{2\rho}
        \} \\
        &= \arg\min\limits_{x} \{
            \frac{1}{2} \lVert y - Hx \rVert_{2}^{2} +
            \frac{\rho}{2} \lVert x-z+\frac{u}{\rho} \rVert_{2}^{2}
        \} \\
    \end{align*}
\end{frame}

\begin{frame}{$\hat x = \arg\min\limits_{x} {L(x, z, u)}$}
    \begin{align*}
        &\frac{d}{dx} (\frac{1}{2} \lVert y - Hx \rVert_{2}^{2} + \lambda P(z) +
        \frac{\rho}{2} \lVert x-z+\frac{u}{\rho} \rVert_{2}^{2} 
        - \frac{\lVert u \rVert_{2}^{2}}{2\rho}) = 0 \\
        &\iff -H^{T} (y - Hx) + \rho (x-z+\frac{u}{\rho}) = 0 \\
        &\iff -H^{T}y + H^{T}Hx + \rho x - \rho z+ u = 0 \\
        &\iff -H^{T}y + (H^{T}H + \rho I)x - \rho z+ u = 0 \\
        &\iff  x = (H^{T}H + \rho I)^{-1} (H^{T}y + \rho z - u) \\
    \end{align*}
\end{frame}

\begin{frame}{$\hat z = \arg\min\limits_{z} {L(x, z, u)}$}
    \begin{align*}
        \hat z
        &= \arg\min\limits_{z} L(x, z, u) \\
        &= \arg\min\limits_{z} \{
            \frac{1}{2} \lVert y - Hx \rVert_{2}^{2} + \lambda P(z) +
            \frac{\rho}{2} \lVert x-z+\frac{u}{\rho} \rVert_{2}^{2} 
            - \frac{\lVert u \rVert_{2}^{2}}{2\rho}
        \} \\
        &= \arg\min\limits_{z} \{
            \lambda P(z) +
            \frac{\rho}{2} \lVert x-z+\frac{u}{\rho} \rVert_{2}^{2} 
        \} \\
        &= \arg\min\limits_{z} \{
            \lambda P(z) +
            \frac{\rho}{2} \lVert -x+z-\frac{u}{\rho} \rVert_{2}^{2} 
        \} \\
        &= \arg\min\limits_{z} \{
            \lambda P(z) +
            \frac{\rho}{2} \lVert -x+z-\frac{u}{\rho} \rVert_{2}^{2} 
        \}
    \end{align*}
\end{frame}

\begin{frame}{Resolution}

    \begin{algorithm}[H]
        \caption{ADMM} % \label{euclid}
        \begin{algorithmic}[1]
            \Procedure{ADMM}{}\newline
            \textbf{Input:} $y$, $\lambda$, $\rho$, $nb\_iterations$ \\
            \textbf{Output:} $x_{nb\_iterations}$
            \For{\texttt{$k \in 0...nb\_iterations$}}
                \State{$x_{k+1} = (H^{T}H + \rho I)^{-1} (H^{T}y + \rho z_{k} - u_{k})$}
                \State{$z_{k+1} = \arg\min\limits_{z} \{
                    \lambda P(z) +
                    \frac{\rho}{2} \lVert -x_{k+1}+z-\frac{u_{k}}{\rho} \rVert_{2}^{2} 
                \}$}
                \State{$u_{k+1} = u_{k} + \rho (z_{k+1} - x_{k+1})$}
            \EndFor
            \EndProcedure
        \end{algorithmic}
        % \label{alg_1}
    \end{algorithm}
    \end{frame}

\begin{frame}{$P(x) = \lVert x \rVert_{1}$}

Opérateur proximal : $\hat{x} = \arg\min\limits_{x} \{ 
    \frac{1}{2} \lVert y - Hx \rVert_{2}^{2} 
    + \lambda {\color{red} \lVert x \rVert_{1} } 
\}$ \\

\vspace{5mm}

Résolution avec l'ADMM:
\begin{itemize}
    \item Équation : $\min\limits_{x, z} \{ 
        \frac{1}{2} \lVert y - Hx \rVert_{2}^{2} 
        + \lambda {\color{red} \lVert z \rVert_{1} } 
    \}$
    \item Contrainte : $x-z = \overrightarrow{0}$
\end{itemize}

\vspace{5mm}

Lagrangien augmenté :

\begin{align*}
L(x, z, u) 
&= \frac{1}{2} \lVert y - Hx \rVert_{2}^{2} + \lambda {\color{red} \lVert z \rVert_{1} }  +
u^{T}(x-z) + \frac{\rho}{2} \lVert x-z \rVert_{2}^{2} \\
&= \frac{1}{2} \lVert y - Hx \rVert_{2}^{2} + \lambda {\color{red} \lVert z \rVert_{1} }  +
\frac{\rho}{2} \lVert x-z+\frac{u}{\rho} \rVert_{2}^{2} - \frac{\lVert u \rVert_{2}^{2}}{2\rho}    
\end{align*}

\end{frame}

\begin{frame}{$P(x) = \lVert x \rVert_{2}^{2}$}

Opérateur proximal : $\hat{x} = \arg \min_{x} \{ 
    \frac{1}{2} \lVert y - Hx \rVert_{2}^{2} 
    + \lambda {\color{red} \lVert x \rVert_{2}^{2}}$ 
\} \\

\vspace{5mm}

Résolution avec l'ADMM:
\begin{itemize}
    \item Équation : $\min_{x, z} \{ 
        \frac{1}{2} \lVert y - Hx \rVert_{2}^{2} 
        + \lambda {\color{red} \lVert x \rVert_{2}^{2}} \}$
    \item Contrainte : $x-z = \overrightarrow{0}$
\end{itemize}

\vspace{5mm}

Lagrangien augmenté :
\begin{align*}
L(x, z, u)
&= \frac{1}{2} \lVert y - Hx \rVert_{2}^{2} 
+ \lambda {\color{red} \lVert z \rVert_{2}^{2}} +
u^{T}(x-z) + \frac{\rho}{2} \lVert x-z \rVert_{2}^{2} \\
&= \frac{1}{2} \lVert y - Hx \rVert_{2}^{2} 
+ \lambda {\color{red} \lVert z \rVert_{2}^{2}} +
\frac{\rho}{2} \lVert x-z+\frac{u}{\rho} \rVert_{2}^{2} - \frac{\lVert u \rVert_{2}^{2}}{2\rho}
\end{align*}

\end{frame}

\begin{frame}{$P(x) = \lVert \nabla x \rVert_{1}$}

Opérateur proximal : $\hat{x} = \arg \min_{x} \{ \frac{1}{2} \lVert y - Hx \rVert_{2}^{2} + \lambda \lVert \nabla x \rVert_{1} \}$ \\

\vspace{5mm}

Résolution avec l'ADMM:
\begin{itemize}
    \item Équation : $\min_{x, z} \{ \frac{1}{2} \lVert y - Hx \rVert_{2}^{2} + \lambda \lVert z \rVert_{1} \}$
    \item Contrainte : $\nabla x - z = \overrightarrow{0}$
\end{itemize}

\vspace{5mm}

Lagrangien augmenté :

\begin{align*}
L(x, z, u) 
&= \frac{1}{2} \lVert y - Hx \rVert_{2}^{2} + \lambda \lVert z \rVert_{1} +
u^{T}(\nabla x-z) + \frac{\rho}{2} \lVert \nabla x-z \rVert_{2}^{2} \\
&= \frac{1}{2} \lVert y - Hx \rVert_{2}^{2} + \lambda \lVert z \rVert_{1} +
\frac{\rho}{2} \lVert \nabla x-z+\frac{u}{\rho} \rVert_{2}^{2} - \frac{\lVert u \rVert_{2}^{2}}{2\rho}    
\end{align*}

\end{frame}

\begin{frame}{Notation $\nabla$}

\begin{align*}
\nabla = \begin{bmatrix} \frac{d}{dx} \\ \frac{d}{dy} \end{bmatrix}
\end{align*}

\begin{align*}
\nabla f 
= \begin{bmatrix} \frac{d}{dx} \\ \frac{d}{dy} \end{bmatrix} f 
= \begin{bmatrix} \frac{df}{dx} \\ \frac{df}{dy}
\end{bmatrix}
\end{align*}

\begin{align*}
\nabla^{T} f 
= \begin{bmatrix} \frac{d}{dx} \\ \frac{d}{dy} \end{bmatrix}^T f
= \begin{bmatrix} \frac{d}{dx} & \frac{d}{dy} \end{bmatrix} f
= \begin{bmatrix} \frac{df}{dx} & \frac{df}{dy} \end{bmatrix}
\end{align*}

\begin{align*}
\nabla^{T} \nabla f = \begin{bmatrix} 
\frac{d}{dx} & \frac{d}{dy} 
\end{bmatrix} \begin{bmatrix} 
\frac{d}{dx} \\ 
\frac{d}{dy} 
\end{bmatrix} f 
= (\frac{d^{2}}{dx^{2}} + \frac{d^{2}}{dy^{2}}) f
= \frac{d^{2}f}{dx^{2}} + \frac{d^{2}f}{dy^{2}}
= \Delta f
\end{align*}
\end{frame}

\begin{frame}{Notation $\nabla$}

\begin{align*}
\nabla . v = \begin{bmatrix} \frac{d}{dx} \\ \frac{d}{dy} \end{bmatrix} .
\begin{bmatrix} v_{x} \\ v_{y} \end{bmatrix}
= \frac{dv_{x}}{dx} + \frac{dv_{y}}{dy}
\end{align*}
\end{frame}




    \section{ADMM}

\frame{\sectionpage}

% \input{chapters/1-Generality/1-ADMM.tex}
% \subsection{Algorithm}

\frame{
    \tableofcontents[ 
        currentsubsection, 
        % hideothersubsections,
        % sectionstyle=show/hide,
        % subsectionstyle=show/shaded,
    ]
}

\subsection{Definition}

\frame{
    \tableofcontents[ 
        currentsubsection, 
        % hideothersubsections,
        % sectionstyle=show/hide,
        % subsectionstyle=show/shaded,
    ]
}

\begin{frame}{Definition}

% Résolution avec Split Bregman:
% \begin{itemize}
%     \item Équation : $\min\limits_{x} \{ f(x) + \lVert \nabla x \rVert_{1} \}$
%     \item Contrainte : $Ax + Bz = c$
% \end{itemize}

% \begin{itemize}
%     \item Équation : $\min\limits_{x} \{ f(x) + \lVert  x \rVert_{1} \}$
%     \item Contrainte : $Ax + Bz = c$
% \end{itemize}

% \vspace{5mm}

% Lagrangien augmenté :

% \begin{align*}
% L(x, z, u) 
% &= f(x) + g(z) + u^{T}(Ax + Bz - c) 
% + \frac{\rho}{2} \lVert Ax + Bz - c \rVert_{2}^{2} \\
% &= f(x) + g(z)
% + \frac{\rho}{2} \lVert Ax + Bz - c + \frac{u}{\rho} \rVert_{2}^{2}
% - \frac{\lVert u \rVert_{2}^{2}}{2\rho} \\
% \end{align*}

\end{frame}
\subsection{Algorithm}

\frame{
    \tableofcontents[ 
        currentsubsection, 
        % hideothersubsections,
        % sectionstyle=show/hide,
        % subsectionstyle=show/shaded,
    ]
}
\subsection{Exemple}

\frame{
    \tableofcontents[ 
        currentsubsection, 
        % hideothersubsections,
        % sectionstyle=show/hide,
        % subsectionstyle=show/shaded,
    ]
}

\begin{frame}{Problème de minimisation}
\begin{align*}
\hat{f} = \arg\min\limits_{f} \{
    \frac{1}{2} 
    \lVert g - Hf \rVert_{2}^{2}
    + \lambda \lVert \nabla f \rVert_{1}
\} 
\end{align*}
\end{frame}

\begin{frame}{Réécriture de $\lVert \nabla f \rVert_{1}$}

En posant, \\
\begin{align*}
\begin{dcases*}
d_x = \nabla_{x}f = \frac{df}{dx} \\
d_y = \nabla_{y}f = \frac{df}{dy} \\
\lVert \nabla f \rVert_{1}
= \lVert (d_x, d_y) \rVert_{1}
= \sum\limits_{i} \sqrt{(d_x)_{i}^{2} + (d_y)_{i}^{2}} \\
\end{dcases*}
\end{align*}

nous avons: \\
\begin{align*}
\hat{f} = \arg\min\limits_{f} \{
    \frac{1}{2} 
    \lVert g - Hf \rVert_{2}^{2}
    + \lambda \lVert (d_x, d_y) \rVert_{1}
\} 
\end{align*}

\end{frame}

\begin{frame}{Réécriture de $\lVert \nabla f \rVert_{1}$ (Relaxation)}
   
nous avons, \\
\resizebox{1.0\linewidth}{!}{
\begin{minipage}{\linewidth}
\begin{align*}
(f, d_{x}, d_{y}) = \arg\min\limits_{f, d_x, d_y} \{
\frac{1}{2} 
\lVert g - Hf \rVert_{2}^{2}
+ \lambda \lVert (d_{x}, d_{y}) \rVert_{1}
+ \frac{\sigma}{2} \lVert d_{x} - \nabla_{x}f \rVert_{2}^{2}
+ \frac{\sigma}{2} \lVert d_{y} - \nabla_{y}f \rVert_{2}^{2}
\}
\end{align*}
\end{minipage}
}

\end{frame}

\begin{frame}{Introduction du processus itératif}

\resizebox{1.0\linewidth}{!}{
\begin{minipage}{\linewidth}
\begin{align*}
(f^{k+1}, d^{k+1}_{x}, d^{k+1}_{y}) = \arg\min\limits_{f, d_x, d_y} \{
\frac{1}{2} 
\lVert g - Hf^{k} \rVert_{2}^{2}
\lambda \lVert (d^{k}_{x}, d^{k}_{y}) \rVert_{1}
+ \frac{\sigma}{2} \lVert d^{k}_{x} - \nabla_{x}f^{k} - b^{k}_{x} \rVert_{2}^{2}
+ \frac{\sigma}{2} \lVert d^{k}_{y} - \nabla_{y}f^{k} - b^{k}_{y} \rVert_{2}^{2}
\}
\end{align*}
\end{minipage}
}

\vspace{1cm}
Avec,
\begin{align*}
\begin{dcases*}
b_{x}^{k+1} = b_{x}^{k} + (\nabla_{x}f^{k+1} - d_{x}^{k+1}) \\
b_{y}^{k+1} = b_{y}^{k} + (\nabla_{y}f^{k+1} - d_{y}^{k+1}) \\
\end{dcases*}
\end{align*}

\end{frame}

\begin{frame}{Division en deux sous-problème}
\begin{enumerate}
    \item Calcul de $f$
    \item Calcul de $(d_x, d_y)$
\end{enumerate}

\end{frame}

\begin{frame}{Calcul de $f$}

\resizebox{1.0\linewidth}{!}{
\begin{minipage}{\linewidth}
\begin{align*}
f^{k+1} &= \arg\min\limits_{f} \{
\frac{1}{2} 
\lVert g - Hf^{k} \rVert_{2}^{2}
+ \lambda \lVert (d^{k}_{x}, d^{k}_{y}) \rVert_{1}
+ \frac{\sigma}{2} \lVert d^{k}_{x} - \nabla_{x}f^{k} - b^{k}_{x} \rVert_{2}^{2}
+ \frac{\sigma}{2} \lVert d^{k}_{y} - \nabla_{y}f^{k} - b^{k}_{y} \rVert_{2}^{2}
\} \\
&= \arg\min\limits_{f} \{
\frac{1}{2} 
\lVert g - SHf^{k} \rVert_{2}^{2}
+ \frac{\sigma}{2} \lVert d^{k}_{x} - \nabla_{x}f^{k} - b^{k}_{x} \rVert_{2}^{2}
+ \frac{\sigma}{2} \lVert d^{k}_{y} - \nabla_{y}f^{k} - b^{k}_{y} \rVert_{2}^{2}
\}
\end{align*}
\end{minipage}
}

\end{frame}

\begin{frame}{Calcul de $f$}

\resizebox{1\linewidth}{!}{
\begin{minipage}{\linewidth}
\begin{align*}
&\frac{d}{df} \{
\frac{1}{2} \lVert g - Hf \rVert_{2}^{2}
+ \frac{\sigma}{2} \lVert d_{x} - \nabla_{x}f - b_{x} \rVert_{2}^{2}
+ \frac{\sigma}{2} \lVert d_{y} - \nabla_{y}f - b_{y} \rVert_{2}^{2}
\} = 0
\end{align*}
\end{minipage}
}

\resizebox{1\linewidth}{!}{
\begin{minipage}{\linewidth}
\begin{align*}
&\implies - H^{T} (g - Hf)
- \sigma \nabla_{x}^{T} (d_{x} - \nabla_{x}f - b_{x})
- \sigma \nabla_{y}^{T} (d_{y} - \nabla_{y}f - b_{y})
= 0 \\
&\iff - H^{T} g + H^{T} Hf
- \sigma \nabla_{x}^{T} d_{x} 
+ \sigma \nabla_{x}^{T} \nabla_{x}f 
+ \sigma \nabla_{x}^{T} b_{x}
- \sigma \nabla_{y}^{T} d_{y} 
+ \sigma \nabla_{y}^{T} \nabla_{y}f 
+ \sigma \nabla_{y}^{T} b_{y}
= 0 \\
&\iff H^{T} Hf
+ \sigma \nabla_{x}^{T} \nabla_{x}f 
+ \sigma \nabla_{y}^{T} \nabla_{y}f 
= \sigma \nabla_{x}^{T} d_{x}
- \sigma \nabla_{x}^{T} b_{x}
+ \sigma \nabla_{y}^{T} d_{y} 
- \sigma \nabla_{y}^{T} b_{y} 
+ H^{T} g \\
\end{align*}
\end{minipage}
}

\begin{columns}

\column{.5\textwidth}
\centering
\resizebox{1\linewidth}{!}{
\begin{minipage}{\linewidth}
\begin{align*}
&H^{T}Hf
+ \sigma \nabla_{x}^{T} \nabla_{x}f 
+ \sigma \nabla_{y}^{T} \nabla_{y}f \\
&= [
    H^{T}H 
    +  \sigma (
            \nabla_{x}^{T} \nabla_{x}
            + \nabla_{y}^{T} \nabla_{y}
        )
] f \\
&= [H^{T}H + \sigma \Delta] f
\end{align*}
\end{minipage}
}

\column{.5\textwidth}
\vline
\centering
\resizebox{1\linewidth}{!}{
\begin{minipage}{\linewidth}
\begin{align*}
&\sigma \nabla_{x}^{T} d_{x}
- \sigma \nabla_{x}^{T} b_{x}
+ \sigma \nabla_{y}^{T} d_{y} 
- \sigma \nabla_{y}^{T} b_{y} 
+ H^{T} g \\
&= \sigma \nabla_{x}^{T} (d_{x} - b_{x})
+ \sigma \nabla_{y}^{T} (d_{y} - b_{y}) 
+ H^{T} g \\
&= \sigma [
    \nabla_{x}^{T} (d_{x} - b_{x})
    + \nabla_{y}^{T} (d_{y} - b_{y})
] + H^{T} g \\
\end{align*}
\end{minipage}
}

\end{columns}
\end{frame}

\begin{frame}{Calcul de $f$}
Ainsi, dans le domaine spatial, nous avons : \\
\resizebox{1\linewidth}{!}{
\begin{minipage}{\linewidth}
\begin{align*}
f = [H^{T}H + \sigma \Delta]^{-1}
[\sigma ( 
    \nabla_{x}^{T} (d_{x} - b_{x})
    + \nabla_{y}^{T} (d_{y} - b_{y})
    ) + H^{T} g] \\
\end{align*}
\end{minipage}
}
\end{frame}

\begin{frame}{Calcul de $(d_x, d_y)$}

\resizebox{1\linewidth}{!}{
\begin{minipage}{\linewidth}
\begin{align*}
(d^{k+1}_{x}, d^{k+1}_{y}) &= \arg\min\limits_{d_x, d_y} \{
\frac{1}{2} 
\lVert g - Hf^{k} \rVert_{2}^{2}
+ \lambda \lVert (d^{k}_{x}, d^{k}_{y}) \rVert_{1}
+ \frac{\sigma}{2} \lVert d^{k}_{x} - \nabla_{x}f^{k} - b^{k}_{x} \rVert_{2}^{2}
+ \frac{\sigma}{2} \lVert d^{k}_{y} - \nabla_{y}f^{k} - b^{k}_{y} \rVert_{2}^{2}
\} \\
&= \arg\min\limits_{d_x, d_y} \{
\lambda \lVert (d^{k}_{x}, d^{k}_{y}) \rVert_{1}
+ \frac{\sigma}{2} \lVert d^{k}_{x} - \nabla_{x}f^{k} - b^{k}_{x} \rVert_{2}^{2}
+ \frac{\sigma}{2} \lVert d^{k}_{y} - \nabla_{y}f^{k} - b^{k}_{y} \rVert_{2}^{2}
\} \\
&= \arg\min\limits_{d_x, d_y} \{
\lambda \lVert (d^{k}_{x}, d^{k}_{y}) \rVert_{1}
+ \frac{\sigma}{2} \lVert d^{k}_{x} - (\nabla_{x}f^{k} + b^{k}_{x}) \rVert_{2}^{2}
+ \frac{\sigma}{2} \lVert d^{k}_{y} - (\nabla_{y}f^{k} + b^{k}_{y}) \rVert_{2}^{2}
\} \\
\end{align*}
\end{minipage}
}

\begin{align*}
\begin{dcases*}
d_{x}^{k+1} = \max (s^{k} - \frac{\lambda}{\sigma}, 0) \frac{s_{x}^{k}}{s^{k}} \\
d_{y}^{k+1} = \max (s^{k} - \frac{\lambda}{\sigma}, 0) \frac{s_{y}^{k}}{s^{k}} \\
\end{dcases*}
\end{align*}

Avec,
\begin{align*}
\begin{dcases*}
s_{x}^{k} = \nabla_{x}f^{k} + b_{x}^{k} \\
s_{y}^{k} = \nabla_{y}f^{k} + b_{y}^{k} \\
s^{k} = \sqrt{(s_{x}^{k})^{2} + (s_{y}^{k})^{2}}
\end{dcases*}
\end{align*}

\end{frame}

% \begin{frame}{Remarques sur la notation $\nabla$ (gradient)}

% % \begin{align*}
% % \nabla 
% % = \begin{bmatrix} \frac{d}{dx} \\ \frac{d}{dy} \end{bmatrix}
% % = \begin{bmatrix} \nabla_{x} \\ \nabla_{y} \end{bmatrix}
% % \end{align*}

% \begin{align*}
% \nabla 
% &= \begin{bmatrix} \frac{d}{dx} \\ \frac{d}{dy} \end{bmatrix}
% = \begin{bmatrix} \nabla_{x} \\ \nabla_{y} \end{bmatrix} \\
% \nabla^{T}
% &= \begin{bmatrix} \frac{d}{dx} \\ \frac{d}{dy} \end{bmatrix}^{T}
% = \begin{bmatrix} \frac{d}{dx} & \frac{d}{dy} \end{bmatrix} \\
% &= \begin{bmatrix} \nabla_{x} \\ \nabla_{y} \end{bmatrix}^{T}
% = \begin{bmatrix} \nabla_{x}^{T} & \nabla_{y}^{T} \end{bmatrix} 
% \text{(Transpose of block matrix)} \\
% \nabla f 
% &= \begin{bmatrix} \frac{d}{dx} \\ \frac{d}{dy} \end{bmatrix} f 
% = \begin{bmatrix} \frac{df}{dx} \\ \frac{df}{dy} \end{bmatrix}
% = \begin{bmatrix} \nabla_{x}f \\ \nabla_{y}f \end{bmatrix}
% \end{align*}


% \end{frame}

% \begin{frame}{Remarques sur la notation $\nabla$ (gradient)}

% % \begin{align*}
% % \nabla 
% % = \begin{bmatrix} \frac{d}{dx} \\ \frac{d}{dy} \end{bmatrix}
% % = \begin{bmatrix} \nabla_{x} \\ \nabla_{y} \end{bmatrix}
% % \end{align*}

% \begin{align*}
% \nabla 
% &= \begin{bmatrix} \frac{d}{dx} \\ \frac{d}{dy} \end{bmatrix}
% = \begin{bmatrix} \nabla_{x} \\ \nabla_{y} \end{bmatrix} \\
% \nabla^{T}
% &= \begin{bmatrix} \frac{d}{dx} \\ \frac{d}{dy} \end{bmatrix}^{T}
% = \begin{bmatrix} \frac{d}{dx} & \frac{d}{dy} \end{bmatrix} \\
% &= \begin{bmatrix} \nabla_{x} \\ \nabla_{y} \end{bmatrix}^{T}
% = \begin{bmatrix} \nabla_{x}^{T} & \nabla_{y}^{T} \end{bmatrix} 
% \text{(Transpose of block matrix)} \\
% \nabla f 
% &= \begin{bmatrix} \frac{d}{dx} \\ \frac{d}{dy} \end{bmatrix} f 
% = \begin{bmatrix} \frac{df}{dx} \\ \frac{df}{dy} \end{bmatrix}
% = \begin{bmatrix} \nabla_{x}f \\ \nabla_{y}f \end{bmatrix}
% \end{align*}


% \end{frame}

% \begin{frame}{Remarques sur la notation $\Delta$ (Laplacien)}

% \begin{align*}
% \nabla^{T} \nabla &= \begin{bmatrix} 
% \frac{d}{dx} \\ 
% \frac{d}{dy} 
% \end{bmatrix}^{T}
% \begin{bmatrix} 
% \frac{d}{dx} \\ 
% \frac{d}{dy} 
% \end{bmatrix}
% = \begin{bmatrix} 
% \frac{d}{dx} & \frac{d}{dy} 
% \end{bmatrix} \begin{bmatrix} 
% \frac{d}{dx} \\ 
% \frac{d}{dy} 
% \end{bmatrix}
% = \frac{d^{2}}{dx^{2}} + \frac{d^{2}}{dy^{2}}
% = \Delta \\
% &= \begin{bmatrix} 
% \nabla_{x} \\ 
% \nabla_{y} 
% \end{bmatrix}^{T}
% \begin{bmatrix} 
% \nabla_{x} \\ 
% \nabla_{y} 
% \end{bmatrix}
% = \begin{bmatrix} 
% \nabla_{x}^{T} & \nabla_{y}^{T}
% \end{bmatrix} \begin{bmatrix} 
% \nabla_{x} \\ 
% \nabla_{y} 
% \end{bmatrix}
% = \nabla_{x}^{T} \nabla_{x} + \nabla_{y}^{T} \nabla_{y}
% = \Delta \\
% \end{align*}

% \end{frame}



    % \section{Non Local Means (NLM)}
\frame{\sectionpage}
    
 \begin{frame}{Principle}

\begin{itemize}
    \item Iterative method
    \item Non Linear Filtering
    \item Using knowledge about neighbors
\end{itemize}

\end{frame}

\begin{frame}{Filter Expression}
\only<-2>{

    \uncover<+->{\begin{equation*}
    w(i, j, k, l) = - e^{\frac{\norm{I(i,j)-I(k,l)}^{2}_{2}}{2\sigma^{2}}}
    \end{equation*}}
    
    \uncover<+->{\begin{equation*}
    I_{D}(i,j) = \frac{\sum_{k,l} I(k,l) w(i,j,k,l)}{\sum_{k,l} w(i,j,k,l)}
    \end{equation*}}

}
\end{frame}

\begin{frame}{Algorithm}
\begin{algorithm}[H]
    \caption{Filtering Algorithm} % \label{euclid}
    \begin{algorithmic}[1]
        \Procedure{Denoising With Non Local Means Filter}{}\newline
        \textbf{Input:} $I$, $\sigma$, $(n_w, n_h)$ \\
        \textbf{Output:} $I_{D}$
        \For{\texttt{$pixel \in I$}}
            \State{$neighs = neighboors\_of(pixel, n_w, n_h)$}
            \State{$I_{D}[pixel] = non\_local\_mean(pixel, neighs, \sigma)$}
        \EndFor
        \EndProcedure
    \end{algorithmic}
    % \label{alg_1}
\end{algorithm}
\end{frame}

\begin{frame}{Data}
\centering
\begin{columns}
\column{.5\textwidth}
\centering
\includegraphics[scale=0.25]{images/results/original.png}
\column{.5\textwidth}
\centering
\includegraphics[scale=0.25]{images/results/noised.png}
\end{columns}
\end{frame}

\begin{frame}{Results (with $(n_w, n_h) = (5, 5)$)}
\centering
\begin{columns}
\column{.5\textwidth}
\centering
\includegraphics[scale=0.25]{images/results/nlm/plot_mae.png}
\column{.5\textwidth}
\centering
\includegraphics[scale=0.25]{images/results/nlm/plot_mse.png}
\end{columns}
\end{frame}

\begin{frame}{Results (with $(n_w, n_h) = (5, 5)$)}
\centering
\begin{columns}
\column{.5\textwidth}
\centering
\includegraphics[scale=0.25]{images/results/nlm/image_mae.png}
\column{.5\textwidth}
\centering
\includegraphics[scale=0.25]{images/results/nlm/image_mse.png}
\end{columns}
\end{frame}
    % \section{Bilateral Filter}
\frame{\sectionpage}

\begin{frame}{Principle}

\begin{itemize}
    \item Iterative method
    \item Using knowledges about neighboors
\end{itemize}

\end{frame}

\begin{frame}{Bilateral Filter Expression}
\only<-2>{

    \uncover<+->{\begin{equation*}
    w(i, j, k, l) = e^{-\frac{(i-k)^2+(j-l)^2}{2\sigma_{d}^2} 
    - \frac{\norm{I(i,j)-I(k,l)}^{2}_{2}}{2\sigma_{r}^{2}}}
    \end{equation*}}
    
    \uncover<+->{\begin{equation*}
    I_{D}(i,j) = \frac{\sum_{k,l} I(k,l) w(i,j,k,l)}{\sum_{k,l} w(i,j,k,l)}
    \end{equation*}}

}
\end{frame}

\begin{frame}{Algorithm}
\begin{algorithm}[H]
    \caption{Filtering Algorithm} % \label{euclid}
    \begin{algorithmic}[1]
        \Procedure{DeNoising With Bilateral Filter}{}\newline
        \textbf{Input:} $I$, $\sigma_{d}$, $\sigma_{r}$, $n_w$, $n_h$ \\
        \textbf{Output:} $I_{D}$
        \For{\texttt{$pixel \in I$}}
            \State{$neighs = neighboors\_of(pixel, n_w, n_h)$}
            \State{$I_{D}[pixel] = bilateral\_filter(pixel, neighs, \sigma_{d}, \sigma_{r})$}
        \EndFor
        \EndProcedure
    \end{algorithmic}
    % \label{alg_1}
\end{algorithm}
\end{frame}

\begin{frame}{Results}
\centering
TODO
\end{frame}
    
    % \section{Block-matching and 3D filtering (BM3D)}
\frame{\sectionpage}

\begin{frame}{Principle}

\begin{itemize}
    \item Iterative method
    \item Using knowledges about neighboors
    \item Grouping patches with "similarity" measure
\end{itemize}

\end{frame}
    
    % \input{chapters/fourier-playground.tex}
    
    % \section*{Acknowledgments} %You can remove this if you do not want to use it
    %     \begin{frame}{Acknowledgments}
    %         The author is extremely thankful to Prof. Antônio F. R. T. Piza for the short, yet wonderful, conversations about this seminar.
    %     \end{frame}
    
    % \section*{References} %You can remove this if you do not want to use it
    %     \nocite{Djairo} \nocite{PhilPanof} \nocite{Fleming} \nocite{Shankar}
    %     \begin{frame}{References}
    %         \printbibliography
    %     \end{frame}

    % \section{}
    \begin{frame}{}
        \centering
            \Huge\bfseries
        \textcolor{orange}{The End}
    \end{frame}
\end{document}
